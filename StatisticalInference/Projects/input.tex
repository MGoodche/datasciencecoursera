\documentclass[]{article}
\usepackage{lmodern}
\usepackage{amssymb,amsmath}
\usepackage{ifxetex,ifluatex}
\usepackage{fixltx2e} % provides \textsubscript
\ifnum 0\ifxetex 1\fi\ifluatex 1\fi=0 % if pdftex
  \usepackage[T1]{fontenc}
  \usepackage[utf8]{inputenc}
\else % if luatex or xelatex
  \ifxetex
    \usepackage{mathspec}
  \else
    \usepackage{fontspec}
  \fi
  \defaultfontfeatures{Ligatures=TeX,Scale=MatchLowercase}
\fi
% use upquote if available, for straight quotes in verbatim environments
\IfFileExists{upquote.sty}{\usepackage{upquote}}{}
% use microtype if available
\IfFileExists{microtype.sty}{%
\usepackage{microtype}
\UseMicrotypeSet[protrusion]{basicmath} % disable protrusion for tt fonts
}{}
\usepackage[margin=1in]{geometry}
\usepackage{hyperref}
\hypersetup{unicode=true,
            pdftitle={Statistical Inference Project - Simulation},
            pdfauthor={Miriam Pavón},
            pdfborder={0 0 0},
            breaklinks=true}
\urlstyle{same}  % don't use monospace font for urls
\usepackage{color}
\usepackage{fancyvrb}
\newcommand{\VerbBar}{|}
\newcommand{\VERB}{\Verb[commandchars=\\\{\}]}
\DefineVerbatimEnvironment{Highlighting}{Verbatim}{commandchars=\\\{\}}
% Add ',fontsize=\small' for more characters per line
\usepackage{framed}
\definecolor{shadecolor}{RGB}{248,248,248}
\newenvironment{Shaded}{\begin{snugshade}}{\end{snugshade}}
\newcommand{\KeywordTok}[1]{\textcolor[rgb]{0.13,0.29,0.53}{\textbf{{#1}}}}
\newcommand{\DataTypeTok}[1]{\textcolor[rgb]{0.13,0.29,0.53}{{#1}}}
\newcommand{\DecValTok}[1]{\textcolor[rgb]{0.00,0.00,0.81}{{#1}}}
\newcommand{\BaseNTok}[1]{\textcolor[rgb]{0.00,0.00,0.81}{{#1}}}
\newcommand{\FloatTok}[1]{\textcolor[rgb]{0.00,0.00,0.81}{{#1}}}
\newcommand{\ConstantTok}[1]{\textcolor[rgb]{0.00,0.00,0.00}{{#1}}}
\newcommand{\CharTok}[1]{\textcolor[rgb]{0.31,0.60,0.02}{{#1}}}
\newcommand{\SpecialCharTok}[1]{\textcolor[rgb]{0.00,0.00,0.00}{{#1}}}
\newcommand{\StringTok}[1]{\textcolor[rgb]{0.31,0.60,0.02}{{#1}}}
\newcommand{\VerbatimStringTok}[1]{\textcolor[rgb]{0.31,0.60,0.02}{{#1}}}
\newcommand{\SpecialStringTok}[1]{\textcolor[rgb]{0.31,0.60,0.02}{{#1}}}
\newcommand{\ImportTok}[1]{{#1}}
\newcommand{\CommentTok}[1]{\textcolor[rgb]{0.56,0.35,0.01}{\textit{{#1}}}}
\newcommand{\DocumentationTok}[1]{\textcolor[rgb]{0.56,0.35,0.01}{\textbf{\textit{{#1}}}}}
\newcommand{\AnnotationTok}[1]{\textcolor[rgb]{0.56,0.35,0.01}{\textbf{\textit{{#1}}}}}
\newcommand{\CommentVarTok}[1]{\textcolor[rgb]{0.56,0.35,0.01}{\textbf{\textit{{#1}}}}}
\newcommand{\OtherTok}[1]{\textcolor[rgb]{0.56,0.35,0.01}{{#1}}}
\newcommand{\FunctionTok}[1]{\textcolor[rgb]{0.00,0.00,0.00}{{#1}}}
\newcommand{\VariableTok}[1]{\textcolor[rgb]{0.00,0.00,0.00}{{#1}}}
\newcommand{\ControlFlowTok}[1]{\textcolor[rgb]{0.13,0.29,0.53}{\textbf{{#1}}}}
\newcommand{\OperatorTok}[1]{\textcolor[rgb]{0.81,0.36,0.00}{\textbf{{#1}}}}
\newcommand{\BuiltInTok}[1]{{#1}}
\newcommand{\ExtensionTok}[1]{{#1}}
\newcommand{\PreprocessorTok}[1]{\textcolor[rgb]{0.56,0.35,0.01}{\textit{{#1}}}}
\newcommand{\AttributeTok}[1]{\textcolor[rgb]{0.77,0.63,0.00}{{#1}}}
\newcommand{\RegionMarkerTok}[1]{{#1}}
\newcommand{\InformationTok}[1]{\textcolor[rgb]{0.56,0.35,0.01}{\textbf{\textit{{#1}}}}}
\newcommand{\WarningTok}[1]{\textcolor[rgb]{0.56,0.35,0.01}{\textbf{\textit{{#1}}}}}
\newcommand{\AlertTok}[1]{\textcolor[rgb]{0.94,0.16,0.16}{{#1}}}
\newcommand{\ErrorTok}[1]{\textcolor[rgb]{0.64,0.00,0.00}{\textbf{{#1}}}}
\newcommand{\NormalTok}[1]{{#1}}
\usepackage{graphicx,grffile}
\makeatletter
\def\maxwidth{\ifdim\Gin@nat@width>\linewidth\linewidth\else\Gin@nat@width\fi}
\def\maxheight{\ifdim\Gin@nat@height>\textheight\textheight\else\Gin@nat@height\fi}
\makeatother
% Scale images if necessary, so that they will not overflow the page
% margins by default, and it is still possible to overwrite the defaults
% using explicit options in \includegraphics[width, height, ...]{}
\setkeys{Gin}{width=\maxwidth,height=\maxheight,keepaspectratio}
\IfFileExists{parskip.sty}{%
\usepackage{parskip}
}{% else
\setlength{\parindent}{0pt}
\setlength{\parskip}{6pt plus 2pt minus 1pt}
}
\setlength{\emergencystretch}{3em}  % prevent overfull lines
\providecommand{\tightlist}{%
  \setlength{\itemsep}{0pt}\setlength{\parskip}{0pt}}
\setcounter{secnumdepth}{0}
% Redefines (sub)paragraphs to behave more like sections
\ifx\paragraph\undefined\else
\let\oldparagraph\paragraph
\renewcommand{\paragraph}[1]{\oldparagraph{#1}\mbox{}}
\fi
\ifx\subparagraph\undefined\else
\let\oldsubparagraph\subparagraph
\renewcommand{\subparagraph}[1]{\oldsubparagraph{#1}\mbox{}}
\fi

%%% Use protect on footnotes to avoid problems with footnotes in titles
\let\rmarkdownfootnote\footnote%
\def\footnote{\protect\rmarkdownfootnote}

%%% Change title format to be more compact
\usepackage{titling}

% Create subtitle command for use in maketitle
\newcommand{\subtitle}[1]{
  \posttitle{
    \begin{center}\large#1\end{center}
    }
}

\setlength{\droptitle}{-2em}
  \title{Statistical Inference Project - Simulation}
  \pretitle{\vspace{\droptitle}\centering\huge}
  \posttitle{\par}
  \author{Miriam Pavón}
  \preauthor{\centering\large\emph}
  \postauthor{\par}
  \predate{\centering\large\emph}
  \postdate{\par}
  \date{10/03/2018}


\begin{document}
\maketitle

\subsection{Overview}\label{overview}

This project is to investigate the exponential distribution in R and
compare it with the Central Limit Theorem. The exponential distribution
can be simulated in R with rexp(n, lambda) where lambda is the rate
parameter. The mean of exponential distribution is 1/lambda and the
standard deviation is also 1/lambda.

Click on links here to quickly view tasks completed in this assignment:

\begin{enumerate}
\def\labelenumi{\arabic{enumi}.}
\tightlist
\item
  \protect\hyperlink{point1}{Simulation}
\item
  \protect\hyperlink{point2}{Show the sample mean and compare it to the
  theoretical mean of the distribution.}
\item
  \protect\hyperlink{point3}{Show how variable the sample is (via
  variance) and compare it to the theoretical variance of the
  distribution.}
\item
  \protect\hyperlink{point4}{Show that the distribution is approximately
  normal.}
\end{enumerate}

\subsection{Data}\label{data}

The data for this assignment comes from this:

\begin{Shaded}
\begin{Highlighting}[]
\NormalTok{s <-}\StringTok{ }\DecValTok{1000} \CommentTok{#Number of simulations}
\NormalTok{lambda <-}\StringTok{ }\FloatTok{0.2} \CommentTok{#Rate for all simulations}
\NormalTok{n <-}\StringTok{ }\DecValTok{40} \CommentTok{#Number of samples}
\KeywordTok{set.seed}\NormalTok{(}\DecValTok{240}\NormalTok{) }\CommentTok{#set seed for reproducability}
\end{Highlighting}
\end{Shaded}

\subsection{Requirements}\label{requirements}

For this assignment you will need some specific tools

\emph{} RStudio: \emph{} You will need RStudio to publish your completed
analysis document to RPubs. You can also use RStudio to edit/write your
analysis.

\emph{} knitr:\emph{} You will need the knitr package in order to
compile your R Markdown document and convert it to HTML

Before beginning the project, be sure to load the required R libraries
and set any environmental variables. Note that setting messages in
markdown to false suppresses messages from library loading such as
version number and dependencies. Updating to latest versions of these
libraries may improve ability to obtain results fairly similar to the
steps outlined here.

\begin{Shaded}
\begin{Highlighting}[]
\CommentTok{# load libraries}
\KeywordTok{library}\NormalTok{(ggplot2)}
\KeywordTok{library}\NormalTok{(knitr)}
\end{Highlighting}
\end{Shaded}

\subsubsection{1. Simulations}\label{simulations}

\begin{Shaded}
\begin{Highlighting}[]
\NormalTok{#### 1. Simulation}

\NormalTok{simulation <-}\StringTok{ }\KeywordTok{replicate}\NormalTok{(s, }\KeywordTok{mean}\NormalTok{(}\KeywordTok{rexp}\NormalTok{(n,lambda)))}

\NormalTok{simulation_mean <-}\StringTok{ }\KeywordTok{mean}\NormalTok{(simulation) }\CommentTok{#Distribution Mean}
\NormalTok{simulation_mean}
\end{Highlighting}
\end{Shaded}

\begin{verbatim}
## [1] 5.026972
\end{verbatim}

\begin{Shaded}
\begin{Highlighting}[]
\NormalTok{simulation_median <-}\StringTok{ }\KeywordTok{median}\NormalTok{(simulation) }\CommentTok{#Median}
\NormalTok{simulation_median}
\end{Highlighting}
\end{Shaded}

\begin{verbatim}
## [1] 4.997169
\end{verbatim}

\begin{Shaded}
\begin{Highlighting}[]
\NormalTok{simulation_sd     <-}\StringTok{ }\KeywordTok{sd}\NormalTok{(simulation) }\CommentTok{#Standard Deviation}
\NormalTok{simulation_sd }
\end{Highlighting}
\end{Shaded}

\begin{verbatim}
## [1] 0.8111959
\end{verbatim}

\protect\hyperlink{top}{Back to top}

\subsubsection{2. Show the sample mean and compare it to the theoretical
mean of the
distribution.}\label{show-the-sample-mean-and-compare-it-to-the-theoretical-mean-of-the-distribution.}

The simulation and the theory means are the next ones:

\begin{Shaded}
\begin{Highlighting}[]
\NormalTok{simulation_mean  <-}\StringTok{ }\KeywordTok{mean}\NormalTok{(simulation) }\CommentTok{#Distribution Mean}
\NormalTok{simulation_mean}
\end{Highlighting}
\end{Shaded}

\begin{verbatim}
## [1] 5.026972
\end{verbatim}

\begin{Shaded}
\begin{Highlighting}[]
\NormalTok{theoretical_mean  <-}\StringTok{ }\DecValTok{1}\NormalTok{/lambda }\CommentTok{#Analytical Mean}
\NormalTok{theoretical_mean}
\end{Highlighting}
\end{Shaded}

\begin{verbatim}
## [1] 5
\end{verbatim}

The actual center of the distribution of the average of 40 exponetialsis
very close to its theoretical center of the distribution.

This is code for the graphical representation:

\begin{Shaded}
\begin{Highlighting}[]
\KeywordTok{hist}\NormalTok{(simulation, }\DataTypeTok{xlab =} \StringTok{"Mean"}\NormalTok{, }\DataTypeTok{main =} \StringTok{"Histogram of 1000 means of 40 sample exponentials"}\NormalTok{)}
\KeywordTok{abline}\NormalTok{(}\DataTypeTok{v =} \NormalTok{simulation_mean, }\DataTypeTok{col =} \StringTok{"green"}\NormalTok{,}\DataTypeTok{lwd=}\DecValTok{3}\NormalTok{, }\DataTypeTok{lty=}\DecValTok{2}\NormalTok{)}
\KeywordTok{abline}\NormalTok{(}\DataTypeTok{v =} \NormalTok{theoretical_mean, }\DataTypeTok{col =} \StringTok{"red"}\NormalTok{,}\DataTypeTok{lwd=}\DecValTok{1}\NormalTok{, }\DataTypeTok{lty=}\DecValTok{1}\NormalTok{)}
\KeywordTok{legend}\NormalTok{(}\StringTok{'topright'}\NormalTok{, }\KeywordTok{c}\NormalTok{(}\StringTok{"Theoretical Mean"}\NormalTok{,}\StringTok{"Sample Mean"}\NormalTok{), }
       \DataTypeTok{col=}\KeywordTok{c}\NormalTok{(}\StringTok{"red"}\NormalTok{, }\StringTok{"green"}\NormalTok{), }\DataTypeTok{lty=}\KeywordTok{c}\NormalTok{(}\DecValTok{1}\NormalTok{,}\DecValTok{1}\NormalTok{), }\DataTypeTok{bty =} \StringTok{"n"}\NormalTok{)}
\end{Highlighting}
\end{Shaded}

\includegraphics{StatisticalInferenceProject_files/figure-latex/unnamed-chunk-5-1.pdf}

\protect\hyperlink{top}{Back to top}

\begin{center}\rule{0.5\linewidth}{\linethickness}\end{center}

\subsubsection{3. Show how variable the sample is (via variance) and
compare it to the theoretical variance of the
distribution.}\label{show-how-variable-the-sample-is-via-variance-and-compare-it-to-the-theoretical-variance-of-the-distribution.}

\begin{Shaded}
\begin{Highlighting}[]
\CommentTok{# Theoretical variance}
\NormalTok{theoretical_sd <-}\StringTok{ }\NormalTok{(}\DecValTok{1}\NormalTok{/lambda)/}\KeywordTok{sqrt}\NormalTok{(n)}
\NormalTok{theoretical_variance <-}\StringTok{ }\NormalTok{theoretical_sd^}\DecValTok{2}
\NormalTok{theoretical_variance}
\end{Highlighting}
\end{Shaded}

\begin{verbatim}
## [1] 0.625
\end{verbatim}

\begin{Shaded}
\begin{Highlighting}[]
\CommentTok{# Simulated variance}
\NormalTok{simulated_variance  <-}\StringTok{ }\KeywordTok{var}\NormalTok{(simulation)}
\NormalTok{simulated_variance}
\end{Highlighting}
\end{Shaded}

\begin{verbatim}
## [1] 0.6580389
\end{verbatim}

So, as we can see, both variances are quite close to each other.

\protect\hyperlink{top}{Back to top}

\begin{center}\rule{0.5\linewidth}{\linethickness}\end{center}

\subsubsection{4. Show that the distribution is approximately
normal.}\label{show-that-the-distribution-is-approximately-normal.}

\begin{Shaded}
\begin{Highlighting}[]
\NormalTok{xfit <-}\StringTok{ }\KeywordTok{seq}\NormalTok{(}\KeywordTok{min}\NormalTok{(simulation), }\KeywordTok{max}\NormalTok{(simulation), }\DataTypeTok{length=}\DecValTok{100}\NormalTok{)}
\NormalTok{yfit <-}\StringTok{ }\KeywordTok{dnorm}\NormalTok{(xfit, theoretical_mean , theoretical_sd)}
\KeywordTok{hist}\NormalTok{(simulation,}\DataTypeTok{breaks=}\NormalTok{n,}\DataTypeTok{prob=}\NormalTok{T,}\DataTypeTok{xlab =} \StringTok{"Means"}\NormalTok{,}\DataTypeTok{main=}\StringTok{"Density of means"}\NormalTok{,}\DataTypeTok{ylab=}\StringTok{"Density"}\NormalTok{)}
\KeywordTok{lines}\NormalTok{(xfit, yfit, }\DataTypeTok{pch=}\DecValTok{22}\NormalTok{, }\DataTypeTok{col=}\StringTok{"green"}\NormalTok{, }\DataTypeTok{lwd=}\DecValTok{3}\NormalTok{,}\DataTypeTok{lty=}\DecValTok{5}\NormalTok{)}
\end{Highlighting}
\end{Shaded}

\includegraphics{StatisticalInferenceProject_files/figure-latex/unnamed-chunk-7-1.pdf}

The distribution is approximately normal.

\protect\hyperlink{top}{Back to top}

\begin{center}\rule{0.5\linewidth}{\linethickness}\end{center}


\end{document}
